% Typical usage (all UPPERCASE items are optional):
%       \input 580pre
%       \begin{document}
%       \MYTITLE{Title of document, e.g., Lab 1\\Due ...}
%       \MYHEADERS{short title}{other running head, e.g., due date}
%       \PURPOSE{Description of purpose}
%       \SUMMARY{Very short overview of assignment}
%       \DETAILS{Detailed description}
%         \SUBHEAD{if needed} ...
%         \SUBHEAD{if needed} ...
%          ...
%       \HANDIN{What to hand in and how}
%       \begin{checklist}
%       \item ...
%       \end{checklist}
% There is no need to include a "\documentstyle."
% However, there should be an "\end{document}."
%
%===========================================================
\documentclass[11pt,twoside,titlepage]{article}
%%NEED TO ADD epsf!!
\usepackage{threeparttop}
\usepackage{graphicx}
\usepackage{latexsym}
\usepackage{color}
\usepackage{listings}
\usepackage{fancyvrb}
%\usepackage{pgf,pgfarrows,pgfnodes,pgfautomata,pgfheaps,pgfshade}
\usepackage{tikz}
\usepackage[normalem]{ulem}
\tikzset{
    %Define standard arrow tip
%    >=stealth',
    %Define style for boxes
    oval/.style={
           rectangle,
           rounded corners,
           draw=black, very thick,
           text width=6.5em,
           minimum height=2em,
           text centered},
    % Define arrow style
    arr/.style={
           ->,
           thick,
           shorten <=2pt,
           shorten >=2pt,}
}
\usepackage[noend]{algorithmic}
\usepackage[noend]{algorithm}
\newcommand{\bfor}{{\bf for\ }}
\newcommand{\bthen}{{\bf then\ }}
\newcommand{\bwhile}{{\bf while\ }}
\newcommand{\btrue}{{\bf true\ }}
\newcommand{\bfalse}{{\bf false\ }}
\newcommand{\bto}{{\bf to\ }}
\newcommand{\bdo}{{\bf do\ }}
\newcommand{\bif}{{\bf if\ }}
\newcommand{\belse}{{\bf else\ }}
\newcommand{\band}{{\bf and\ }}
\newcommand{\breturn}{{\bf return\ }}
\newcommand{\mod}{{\rm mod}}
\renewcommand{\algorithmiccomment}[1]{$\rhd$ #1}
\newenvironment{checklist}{\par\noindent\hspace{-.25in}{\bf Checklist:}\renewcommand{\labelitemi}{$\Box$}%
\begin{itemize}}{\end{itemize}}
\pagestyle{threepartheadings}
\usepackage{url}
\usepackage{wrapfig}
% removing the standard hyperref to avoid the horrible boxes
%\usepackage{hyperref}
\usepackage[hidelinks]{hyperref}
% added in the dtklogos for the bibtex formatting
\usepackage{dtklogos}
%=========================
% One-inch margins everywhere
%=========================
\setlength{\topmargin}{0in}
\setlength{\textheight}{8.5in}
\setlength{\oddsidemargin}{0in}
\setlength{\evensidemargin}{0in}
\setlength{\textwidth}{6.5in}
%===============================
%===============================
% Macro for document title:
%===============================
\newcommand{\MYTITLE}[1]%
   {\begin{center}
     \begin{center}
     \bf
     CMPSC 112\\Introduction to Computer Science II\\
     Spring 2014
     \medskip
     \end{center}
     \bf
     #1
     \end{center}
}
%================================
% Macro for headings:
%================================
\newcommand{\MYHEADERS}[2]%
   {\lhead{#1}
    \rhead{#2}
    %\immediate\write16{}
    %\immediate\write16{DATE OF HANDOUT?}
    %\read16 to \dateofhandout
    \def \dateofhandout {January 15, 2015}
    \lfoot{\sc Handed out on \dateofhandout}
    %\immediate\write16{}
    %\immediate\write16{HANDOUT NUMBER?}
    %\read16 to\handoutnum
    \def \handoutnum {2}
    \rfoot{Handout \handoutnum}
   }

%================================
% Macro for bold italic:
%================================
\newcommand{\bit}[1]{{\textit{\textbf{#1}}}}

%=========================
% Non-zero paragraph skips.
%=========================
\setlength{\parskip}{1ex}

%=========================
% Create various environments:
%=========================
\newcommand{\PURPOSE}{\par\noindent\hspace{-.25in}{\bf Purpose:\ }}
\newcommand{\SUMMARY}{\par\noindent\hspace{-.25in}{\bf Summary:\ }}
\newcommand{\DETAILS}{\par\noindent\hspace{-.25in}{\bf Details:\ }}
\newcommand{\HANDIN}{\par\noindent\hspace{-.25in}{\bf Hand in:\ }}
\newcommand{\SUBHEAD}[1]{\bigskip\par\noindent\hspace{-.1in}{\sc #1}\\}
%\newenvironment{CHECKLIST}{\begin{itemize}}{\end{itemize}}


\usepackage[compact]{titlesec}

\begin{document}
\MYTITLE{Laboratory Assignment One: A Primer on Version Control and Java Programming}
\MYHEADERS{Laboratory Assignment One}{Due: September 12, 2016}

\section*{Introduction}

Practicing software developers normally use a version control system to manage most of the artifacts produced during the
phases of the software development life cycle.  In this course, we will always use the Git version control system to
manage the files associated with our laboratory assignments.  In this laboratory assignment, you will learn how to use
the Bitbucket service for managing Git repositories and the {\tt git} command-line tool in the Ubuntu Linux operating
system. After connecting to the course's Git repository and creating your own repository, you will compile and run
several Java programs, write about their code and output, and commit your final report to a repository. Finally,
you will practice the completion of a laboratory assignment while working in a team.

\section*{Reading Assignment}

To start your review of the key features of the Java programming language, please study the material in Chapter 1 of
the textbook, paying particularly close attention to the example program in Section 1.7. In addition, please review the
slides that we have discussed during our recent class sessions. If you have questions about this reading assignment or
the material that was presented in class, then please see the course instructor or a teaching assistant. If done
appropriately under the bounds of the Honor Code, you may also post your question to our Slack team.

\section*{Configuring Git and Bitbucket}

During this laboratory assignment and subsequent assignments, we will securely communicate with the Bitbucket.org
servers that will host all of our projects.  In this laboratory assignment, we will perform all of the steps to
configure the accounts on the departmental servers and the Bitbucket service.  Throughout the assignment, you should
refer to the following web site for additional information:
\url{https://confluence.atlassian.com/display/BITBUCKET/Bitbucket+101}.  As you will be required to turn in a report
describing each step that you finish in this assignment, please be sure to keep a record of all of the steps that you
complete and the challenges that you face.  You are also responsible for working with a team of two or three people to
ensure that each member of the team is able to successfully complete each of the steps outlined in this assignment. For
this laboratory assignment, you may pick your own partner (or, team members).

\begin{enumerate}
  \setlength{\itemsep}{0pt}

\item If you do not already have a Bitbucket account, please go to the Bitbucket web site and create one ---
  make sure that you use your {\tt allegheny.edu} email address so that you can create an unlimited number of free
  Bitbucket repositories. Then, upload your ssh key to Bitbucket.

\item If you have never done so before, you must use the {\tt ssh-keygen} program to create secure-shell keys that you
  can use to support your communication with the Bitbucket servers. Follow the prompts to create your keys and save
  them in the default directory (press ``Enter'' after you are prompted: ``{\tt Enter file in which to save the key ...
  :}'', then press ``Enter'' twice if you do not wish to create a passphrase at this time or type your selected
  passphrase if you do).   Type {\tt man ssh-keygen} and talk with the course instructor to learn more about this
  program.  What files does {\tt ssh-keygen} produce?  Where does this program store these files by default?  You do
  not need to complete this step if you have already run {\tt ssh-keygen}.

  Once you have created your ssh keys, you should raise your hand to invite the course instructor to help you with the
  next steps, thus best ensuring that Bitbucket is configured correctly. First, you must log into Bitbucket and look
  in the right corner for an account avatar with a down arrow.  Click on this blue link and then select the ``Manage
  account'' option. Now, scroll down until you found the ``SSH keys'' option and upload your ssh key to Bitbucket. You
  can copy your SSH key by going to the terminal and typing ``{\tt cat \textasciitilde{}/.ssh/id\_rsa.pub}'' command.
  If you cannot complete any of these steps, then please see the instructor. Students who have already created
  and configured their Bitbucket accounts will not need to complete this step unless you did not use your
  {\tt allegheny.edu} email account's name for Bitbucket.

\item Now, you need to test to see if you can authenticate with the Bitbucket servers.  First, show the course
  instructor that you have correctly configured your Bitbucket account.  Now, ask the instructor to share the course's
  Git repository with you.  Open a terminal window on your workstation and change into the directory where you will
  store your files for this course.  For instance, you might make a {\tt cs112F2016/} directory that will contain the
  Git repository that I will always use to share files with you.  Once you have done so, please type the following
  command: ``{\tt git clone git@bitbucket.org:gkapfham/cs112f2016-share.git}''.  If everything worked correctly, you
  should be able to download all of the files that you will need to use for this laboratory assignment. Please resolve
  any problems that you encountered by first reviewing the Bitbucket documentation and then discussing the matter
  with your classmates. If you are still not able to run ``{\tt git clone}'', then please see the instructor.

\item Using your terminal window, you should browse the files that are in this Git repository.  In particular, please
  look in the {\tt labs/lab1/src/} directory and use Vim to study the three Java programs that you find.  Remember, the
  ``{\tt cd}'' command allows you to change into a directory.

\item If you have not done so already, you should reconfigure your Vim text editor so that it uses the ``spf13-vim''
  configuration so that you have many of the advanced editing features on which we will rely this semester. Those
  students who have not installed spf13-vim can follow the instructions at \url{http://vim.spf13.com/} to better
  configure Vim. If you have already installed this Vim configuration, then please make sure that you know how to use
  the plugins and features that it provides; see the course instructor if you are stuck on this step. Please note that
  students in this course are not required to use ``spf13-vim'' if they already have a suitable Vim configuration.
  But, if you do not have a suitable configuration that provides access to advanced Vim plugins, then you are
  encouraged to try ``spf13-vim'' or some other configuration of Vim so that you can easily use a wide variety of
  advanced plugins.

\end{enumerate}

\vspace*{-.1in}

\section*{Creating a New Repository}

Now that you have learned how to clone an existing Git repository, you should make a new repository in the {\tt
  cs112F2016/} directory that you previously created.  First, make a new directory called {\tt cs112F2016-<your user
name>}. Then, change into this directory and type the command ``{\tt git init .}''.  At this point, you should go into
the {\tt cs112f2016-share} repository and use the ``{\tt cp -r}'' command to copy the entire {\tt labs/} directory from
the {\tt cs112f2016-share} repository to {\tt cs112F2016-<your user name>}.  Once the files are in your own Git
repository, please use the ``{\tt git add}'' and ``{\tt git commit}'' commands to add them correctly. If you do not know
how to use ``{\tt git add}'' and ``{\tt git commit}'' in the terminal window, please learn more about them by searching
on the Internet, talking about them with your classmates, and discussing them with the course instructor.

Next, you should use the Bitbucket web site to create a repository that has the same name as the local directory and
local repository.  You must follow Bitbucket's instructions to push the code and tags in your local repository to the
remote one. When you are finished with this step, you should see in your web browser that the Bitbucket servers are
correctly storing the three Java programs. Once the Git repository contains the correct files, you should share your
Bitbucket repository with the course instructor --- my Bitbucket user name is ``gkapfham''.

You can learn more about Git by consulting web sites like  \url{http://gitimmersion.com/}.  At minimum, you should
ensure that you understand how to use the following commands: ``{\tt git init}'', ``{\tt git status}'', ``{\tt git
add}'', ``{\tt git commit}'', ``{\tt git push}'', and ``{\tt git pull}''. If you want to interact with your Git
repository through their Vim text editor---instead of through the terminal window---then you are encourage to learn how
to use the Fugitive plugin that is part of spf13-vim.

\section*{Compiling, Running, and Understanding Java Programs}

Once you have mastered the use of Git and version control, you should return to the {\tt labs/lab1/src/} directory
that contains the two Java programs. Now, use the Java compiler to compile the {\tt Hooray.java} program.  That is,
you should type ``{\tt javac Hooray.java}'' in the terminal window.  Next, you can run this program by typing ``{\tt
java Hooray}'' in the terminal window.  What output does this program produce?  Why does it create this output? How
do you stop this program?

After compiling, using, and studying the {\tt Hooray} program, you should complete the same steps for the {\tt
Weeee.java} program. Go ahead and compile and run this program. What output does it produce? Why does it create this
output? How is the output similar to and different from that which was created by the {\tt Hooray} program? Once you
have finished studying and understanding these two programs, add comments to the code to explain what they do and how
they work. When commenting the source code of these programs, please make sure that you follow the Javadoc standard
explained in Section 1.9.3 of your textbook. Finally, make sure that the commented version of each program is correctly
committed to your Git version control repository hosted by Bitbucket and you have answered the below prompts about your
experiences with these programs.

Now, return to the {\tt cs112F2016-share} repository and find the {\tt CreditCard.java} program that is provided in
Section 1.7 of your textbook. Please compile and run this program, just as you did for the previous two Java programs.
Next, you and your team member(s) should check that the program produces the same output as is given in Code Fragment
1.7 of your textbook. You will note that if you try to run this program by typing ``{\tt java CreditCard}'' in your
terminal window it will not work correctly and instead produce the error message like ``{\tt Error: Could not find or
load main class CreditCard}''. Why does the Java virtual machine produce this error message?

To learn more about what causes this error message, you and your team member(s) should review Section 1.8 of the
textbook, focusing on how Java allows programmers to create Java classes that are contained in a package. What are the
benefits associated with using packages in our Java programs? What is the package that contains the {\tt
CreditCard.java} program? Once you understand this feature of the Java programming language, you should restructure the
way that you save the {\tt CreditCard.java} program on the file system and develop a new command-line that you can use
to run it in the terminal window. Finally, you should confirm that the program produces the same output as is reported
in the textbook. Please see the course instructor or a teaching assistant if you are struggling with the completion of
this step of the laboratory assignment.

\section*{Summary of the Required Deliverables}

This assignment invites you to submit both a printed and an electronic version of these deliverables:

\vspace*{-.05in}
\begin{enumerate}

  \itemsep 0em
  \item A description of the steps that a user must take to configure Git and Bitbucket.
  \item A description of the inputs, outputs, and behavior of the six aforementioned Git commands.
  \item A commented version of the {\tt Hooray.java} and {\tt Weeee.java} programs.

  \item A one-page report that responds to these five prompts about the first two programs:

    \vspace*{-.05in}
    \begin{enumerate}
      \itemsep 0em

      \item The steps that you took to compile and run both of these programs.

      \item A snapshot of the output that each of these programs produce.

      \item An explanation for why these programs create the output that they do.

      \item A discussion of the similarities and differences between these programs and their output.

      \item An explanation of why it is useful to create comments with the Javadoc standard.

    \end{enumerate}

  \item A one-page report that responds to these four prompts about the third program:

    \vspace*{-.05in}
    \begin{enumerate}
      \itemsep 0em

      \item The steps that you took to compile and run this final program.

      \item The output that this program produces in the terminal window.

      \item An explanation of why it is useful to organize Java programs into packages.

      \item A response to the following questions about programming in the Java language:

        \begin{enumerate}
          \itemsep 0em

          \item The {\tt CreditCard} constructor uses the {\tt this} keyword --- what is its purpose?

          \item The {\tt CreditCard} class contains a {\tt static} method --- what does this do?

          \item The {\tt CreditCard} class has a {\tt main} method --- how does it use this method?

          \item How does the {\tt CreditCard} class use a {\tt for} loop and a {\tt while} loop?

          \item The {\tt CreditCard} class creates an array called {\tt wallet} --- what is an array?

        \end{enumerate}

    \end{enumerate}

\end{enumerate}

\vspace*{-.05in}

Before you turn in this assignment, you also must ensure that the course instructor has read access to your Bitbucket
repository that is named according to the convention {\tt cs112F2016-<your user name>}. Please note that each student in
the class is responsible for completing and submitting their own version of this assignment. While it is acceptable for
members of this class to have high-level conversations, you should not share source code or full command lines with your
classmates.  That is, it is necessary to distinguish carefully between the student who discusses the principles
underlying a problem with others and the student who produces assignments that are identical to, or merely variations
on, someone else's work.  Source code or output that is largely similar to other submissions or to online material will
be judged as evidence of violating the Honor Code. Please see the course instructor if you have questions about the
policies for this laboratory assignment.

\end{document}
