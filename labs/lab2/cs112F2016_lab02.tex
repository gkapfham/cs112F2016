\input{labspre.tex}

\usepackage[compact]{titlesec}

\begin{document}
\MYTITLE{Laboratory Assignment Two: Performance Evaluation of Using Arrays and Iteration}
\MYHEADERS{Laboratory Assignment Two}{Due: September 19, 2016}

\section*{Introduction}

The Java programming language provides facilities for you to store many values of the same type and to ``iterate''
through those values using iteration constructs like a {\tt for} loop.

\section*{Reading Assignment}

To start your review of the key features of the Java programming language, please study the material in Chapter 1 of
the textbook, paying particularly close attention to the example program in Section 1.7. In addition, please review the
slides that we have discussed during our recent class sessions. If you have questions about this reading assignment or
the material that was presented in class, then please see the course instructor or a teaching assistant. If done
appropriately under the bounds of the Honor Code, you may also post your question to our Slack team.

\section*{Compiling, Running, and Repairing a Java Program}



\section*{Summary of the Required Deliverables}

This assignment invites you to submit both a printed and an electronic version of these deliverables:

\vspace*{-.05in}
\begin{enumerate}

  \itemsep 0em
  \item Using the Javadoc standard, a commented version of the three Java classes.

  \item The output from running the final version of the {\tt SentencesReverser} program.

  \item A one-page report that responds to these prompts about the provided Java classes:

    \vspace*{-.05in}
    \begin{enumerate}
      \itemsep 0em

      \item What is the mistake that you found and fixed in the three provided Java classes?

      \item What does an instance of the {\tt Sentence} class look like in the computer's memory?


    \end{enumerate}

\end{enumerate}

\vspace*{-.05in}

Before you turn in this assignment, you also must ensure that the course instructor has read access to your Bitbucket
repository that is named according to the convention {\tt cs112F2016-lab01-<user name one>-<user name two>} (you can
adjust the name for a team of three students). Please note that each team in the class is responsible for completing and
submitting its own version of this assignment. While it is acceptable for members of this class to have high-level
conversations, you should not share source code or full command lines with classmates in other teams. That is, it is
necessary to distinguish carefully between the team that discusses the principles underlying a problem with others and
the team that produces assignments that are identical to, or merely variations on, someone else's work.  Source code or
output that is largely similar to other submissions or to online material will be judged as evidence of violating
Allegheny College's Honor Code. Please see the course instructor if you have questions about the policies for this
laboratory assignment.

\end{document}
