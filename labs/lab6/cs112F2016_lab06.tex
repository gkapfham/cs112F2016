% Typical usage (all UPPERCASE items are optional):
%       \input 580pre
%       \begin{document}
%       \MYTITLE{Title of document, e.g., Lab 1\\Due ...}
%       \MYHEADERS{short title}{other running head, e.g., due date}
%       \PURPOSE{Description of purpose}
%       \SUMMARY{Very short overview of assignment}
%       \DETAILS{Detailed description}
%         \SUBHEAD{if needed} ...
%         \SUBHEAD{if needed} ...
%          ...
%       \HANDIN{What to hand in and how}
%       \begin{checklist}
%       \item ...
%       \end{checklist}
% There is no need to include a "\documentstyle."
% However, there should be an "\end{document}."
%
%===========================================================
\documentclass[11pt,twoside,titlepage]{article}
%%NEED TO ADD epsf!!
\usepackage{threeparttop}
\usepackage{graphicx}
\usepackage{latexsym}
\usepackage{color}
\usepackage{listings}
\usepackage{fancyvrb}
%\usepackage{pgf,pgfarrows,pgfnodes,pgfautomata,pgfheaps,pgfshade}
\usepackage{tikz}
\usepackage[normalem]{ulem}
\tikzset{
    %Define standard arrow tip
%    >=stealth',
    %Define style for boxes
    oval/.style={
           rectangle,
           rounded corners,
           draw=black, very thick,
           text width=6.5em,
           minimum height=2em,
           text centered},
    % Define arrow style
    arr/.style={
           ->,
           thick,
           shorten <=2pt,
           shorten >=2pt,}
}
\usepackage[noend]{algorithmic}
\usepackage[noend]{algorithm}
\newcommand{\bfor}{{\bf for\ }}
\newcommand{\bthen}{{\bf then\ }}
\newcommand{\bwhile}{{\bf while\ }}
\newcommand{\btrue}{{\bf true\ }}
\newcommand{\bfalse}{{\bf false\ }}
\newcommand{\bto}{{\bf to\ }}
\newcommand{\bdo}{{\bf do\ }}
\newcommand{\bif}{{\bf if\ }}
\newcommand{\belse}{{\bf else\ }}
\newcommand{\band}{{\bf and\ }}
\newcommand{\breturn}{{\bf return\ }}
\newcommand{\mod}{{\rm mod}}
\renewcommand{\algorithmiccomment}[1]{$\rhd$ #1}
\newenvironment{checklist}{\par\noindent\hspace{-.25in}{\bf Checklist:}\renewcommand{\labelitemi}{$\Box$}%
\begin{itemize}}{\end{itemize}}
\pagestyle{threepartheadings}
\usepackage{url}
\usepackage{wrapfig}
% removing the standard hyperref to avoid the horrible boxes
%\usepackage{hyperref}
\usepackage[hidelinks]{hyperref}
% added in the dtklogos for the bibtex formatting
\usepackage{dtklogos}
%=========================
% One-inch margins everywhere
%=========================
\setlength{\topmargin}{0in}
\setlength{\textheight}{8.5in}
\setlength{\oddsidemargin}{0in}
\setlength{\evensidemargin}{0in}
\setlength{\textwidth}{6.5in}
%===============================
%===============================
% Macro for document title:
%===============================
\newcommand{\MYTITLE}[1]%
   {\begin{center}
     \begin{center}
     \bf
     CMPSC 112\\Introduction to Computer Science II\\
     Spring 2014
     \medskip
     \end{center}
     \bf
     #1
     \end{center}
}
%================================
% Macro for headings:
%================================
\newcommand{\MYHEADERS}[2]%
   {\lhead{#1}
    \rhead{#2}
    %\immediate\write16{}
    %\immediate\write16{DATE OF HANDOUT?}
    %\read16 to \dateofhandout
    \def \dateofhandout {January 15, 2015}
    \lfoot{\sc Handed out on \dateofhandout}
    %\immediate\write16{}
    %\immediate\write16{HANDOUT NUMBER?}
    %\read16 to\handoutnum
    \def \handoutnum {2}
    \rfoot{Handout \handoutnum}
   }

%================================
% Macro for bold italic:
%================================
\newcommand{\bit}[1]{{\textit{\textbf{#1}}}}

%=========================
% Non-zero paragraph skips.
%=========================
\setlength{\parskip}{1ex}

%=========================
% Create various environments:
%=========================
\newcommand{\PURPOSE}{\par\noindent\hspace{-.25in}{\bf Purpose:\ }}
\newcommand{\SUMMARY}{\par\noindent\hspace{-.25in}{\bf Summary:\ }}
\newcommand{\DETAILS}{\par\noindent\hspace{-.25in}{\bf Details:\ }}
\newcommand{\HANDIN}{\par\noindent\hspace{-.25in}{\bf Hand in:\ }}
\newcommand{\SUBHEAD}[1]{\bigskip\par\noindent\hspace{-.1in}{\sc #1}\\}
%\newenvironment{CHECKLIST}{\begin{itemize}}{\end{itemize}}


\usepackage[compact]{titlesec}

\begin{document} \MYTITLE{Laboratory Assignment Six: Using Doubling Experiments to Infer Time Complexities}
\MYHEADERS{Laboratory Assignment Six}{Due: October 24, 2016}

\section*{Introduction}

The current module of the course has focused on the importance and purpose of both empirical and analytical evaluations
of algorithm performance. For this laboratory assignment, we will learn how to use a tool, called {\sc ExpOse}, that
leverages the results from successive doubling experiments to infer the ``actual-worst-case'' time complexity of an
algorithm. Using examples provided with {\sc ExpOse}, you will experimentally determine the actual-worst-case time
complexity of several sorting algorithms and two algorithms for determining if there are no duplicate elements in an
array. Ultimately, you work aims to experimentally confirm some of the analytical evaluations of an algorithm provided
in your textbook and in online sources. Throughout this assignment, you will take additional steps towards seeing the
connection between the experimental and analytical evaluation of algorithms. Finally, you will continue to practice the
use of software engineering tools.

\section*{Review Your Textbook}

To do well on this laboratory assignment, you should review the content about sorting an array in Section 3.1.2
(optionally, students may investigate a more advanced analysis of these algorithms by reading Section 9.4.1 of the
textbook). Next, you should carefully review the content in Section 4.1, paying particularly close attention to the
results in Table 4.1 and Figure 4.1. To learn more about the uniqueness detection algorithms studied in this assignment,
please read Section 4.3.3, specifically noting Code Fragments 4.7 and 4.8.  Additionally, you should also examine the
slides that we have discussed during our recent class sessions. If you have questions about this reading assignment or
the material that was presented in class, then please see the course instructor. If done appropriately, you may also
post your question to the {\tt \#laboratory} channel of our Slack team.

\section*{Using and Enhancing the String Experiment Framework}

To start this laboratory assignment, you should return to the {\tt cs112F2016-share} Git repository and type the ``{\tt
git pull}'' command in the terminal window.  Now, you should have a {\tt lab5/} directory that you can explore further.
Once again, please make sure that you can find the source code in this new directory and you understand why the
directories in the assignment are structured the way that they are. Next, you should use GVim to study the source code
in the {\tt build.xml} file.  As in the past assignments, when editing a Java program you can type ``{\tt ant compile}''
in your terminal window and it will compile the Java class and save the bytecode in the correct subdirectories inside of
the {\tt bin/} directory. Please see the course instructor if you cannot get this to work.

After you have carefully reviewed the source code for {\tt StringExperiment.java}, you should compile and run this
program, using the {\tt build.xml} rules that correctly sets the {\tt CLASSPATH} and references the Java program with
its ``fully qualified name''. Please run the {\tt StringExperiment} multiple times and note the data tables that it
produces. You should be aware of the fact that some input sizes for this program may take a very long time to run! If
that is the case, then you may want to reconsider the program's current implementation and change it so that the
experiment runs more efficiently. What trends do you see in this data set? How do your tables of data compare to the one
that the authors present on page 152 of your textbook? Can you clearly explain why these data values are evident in your
data set and the one produced by the textbook's authors?

After you have tried out the {\tt StringExperiment}, you will notice that it does not run the method under study for
multiple trials and then compute summary statistics (i.e., the arithmetic mean and the standard deviation) of the
timings. Therefore, you should improve the program by moving the statistics calculation code from a previous laboratory
assignment into your source code repository for this assignment.  Then, you should carefully integrate this new code so
that it runs multiple trials and calculates the arithmetic mean and standard deviation for the numerous timings. Again,
you may want to think about the feasibility of running one or both algorithms for multiple trials and very long strings
because it may take an inordinate amount of time to run the experiment.

It is worth pointing out that your textbook contains several useful insights into the pattern that you should follow
when making observations about time overhead. For instance, when describing the results from running the {\tt
StringExperiment}, page 153 notes that ``[A]s the value of $n$ is doubled, the running time of {\tt repeat1} typically
increases more than fourfold.'' What does this suggest about the likely worst-case time complexity of the {\tt repeat1}
method? Additionally, page 172 includes the following statement when describing the performance of {\tt repeat2}: ``the
running times in that table $\ldots$ demonstrate a trend of approximately doubling each time the problem size doubles.''
Again, what would this observation suggest about the likely worst-case time complexity of {\tt repeat2}?

Finally, using what you know about the implementation of Strings in the Java language, you should explain why one of the
algorithms has substantially better performance than the other. Ultimately, you should submit the commented code of and
results from using your experimentation framework that systematically doubles the size of the input to the {\tt repeat1} and
{\tt repeat2} methods.

\section*{Carefully Review the Honor Code}

The Academic Honor Program that governs the entire academic program at Allegheny College is described in the Allegheny
Academic Bulletin. The Honor Program applies to all work that is submitted for academic credit or to meet non-credit
requirements for graduation at Allegheny. This includes all work assigned for this class (e.g., examinations, laboratory
assignments, and the final project). All students who have enrolled in the College will work under the Honor Program.

% Each student who has matriculated at the College has acknowledged the following pledge:
% \vspace*{-.1in}
% \begin{quote}
%   I hereby recognize and pledge to fulfill my responsibilities, as defined in the Honor Code, and to maintain the
%   integrity of both myself and the College community as a whole.
% \end{quote}
% \vspace*{-.1in}

% \noindent It is understood that an important part of the learning process in any course, and particularly one in
% computer science, derives from thoughtful discussions with teachers and fellow students.  Such dialogue is encouraged.
% However, it is necessary to distinguish carefully between the student who discusses the principles underlying a problem
% with others and the student who produces assignments that are identical to, or merely variations on, someone else's
% work. While it is acceptable for partners in this class to discuss their programs, data sets, and reports with their
% classmates, deliverables that are nearly identical to the work of others will be taken as evidence of violating the
% \mbox{Honor Code}.

\section*{Summary of the Required Deliverables}

This assignment invites you to submit a signed and printed version of the following deliverables:

\begin{enumerate}

  \itemsep0in

  \item A sample output from running all doubling experiments in all of their relevant configurations.

  \item A description of all of the features supported by your own doubling experiment tool.

  \item Using JavaDoc comments, the documented version of the {\tt StringExperiment} source code.

  \item A comprehensive written report that fully explains the results of your experimental studies.

  \item A reflective commentary on the challenges that you faced when implementing your program.

  \item A reflective commentary on the challenges that you faced when conducting the experiments.

  \item Suggestions for the types of laboratory assignments that you would like to later complete.

\end{enumerate}

Again, along with turning in a printed version of these deliverables, you should ensure that everything is also
available in the repository that is named according to the convention {\tt cs112F2016-<your user name>}. Please remember
that students in the class are responsible for completing and submitting their own version of this assignment. You
should see the instructor if you have any questions.

% While it is acceptable for members of this class to have high-level conversations, you should not share source code or
% full command lines with your classmates.  Deliverables that are nearly identical to the work of others will be taken as
% evidence of violating the \mbox{Honor Code}.  Please see the instructor if you have questions about the policies for
% this assignment.

\end{document}
