\input{labspre.tex}

\usepackage[compact]{titlesec}

\begin{document} \MYTITLE{Laboratory Assignment Eight: Performance Evaluation of a Security Program}
\MYHEADERS{Laboratory Assignment Seven}{Due: October 31, 2016}

\section*{Introduction}

In the past laboratory assignments and classroom discussions we have investigated: (i) algorithms, (ii) data structures,
(iii) the Java programming language, (iv) tools for software engineering, and (v) the analytical and empirical
evaluation of algorithms. In this laboratory assignment, you will specify, design, and implement your own system that
solves an interesting problem in the field of computer security. In particular, we will implement and evaluate a program
that checks and improves passwords. Your program will perform various checks to determine if the user's password is
secure. If one or more of these checks fail, then you will improve the password until all checks pass. You will
intuitively prove the worst-case time complexity an algorithm in your program and then conduct a doubling experiment to
confirm the correctness of your suggested time complexity.

\section*{Review Your Textbook}

To do well on this laboratory assignment, you should review the content in several sections of the textbook. Please
study Sections 1.8 and 1.9 to learn more about packages and import statements in Java and the correct way to engineer
software. Next, you should carefully review all of Chapter 2, paying particularly close attention to Sections 2.1 and
2.5 and their treatment of object-oriented design principles. You should also study all of Chapter 4, focusing on the
examples, featured on pages 176 and 177, that reveal how to prove the worst-case time complexity of an algorithm.
Finally, if your password checking and improvement system requires the use of recursion, then you should review the
material in Chapter 5. Please see the instructor with questions about this content.

\section*{Features of the Computer Security Program}


\vspace*{-.05in}
\section*{Conducting Experiments to Evaluate Efficiency}
\vspace*{-.05in}


\section*{Summary of the Required Deliverables}

This assignment invites you to submit a signed and printed version of the following deliverables:

\vspace*{-.1in}
\begin{enumerate}
  \itemsep0pt
  \item Using diagram(s), an explanation of how recursion works in the Java programming language.

  \item A short discussion of the different primitive types that are available in the Java language.

  \item The final version of the commented source code for your Fibonacci benchmarking framework.

  \item A comprehensive written report that fully explains the results of your experimental study.

  \item A reflective commentary on the challenges that you faced when enhancing the benchmarks.

  \item A reflective commentary on the challenges that you faced when conducting the experiments.

\end{enumerate}
\vspace*{-.1in}

Along with turning in a printed version of these deliverables, you should ensure that everything is also available in
the repository that is named according to the convention {\tt cs112F2016-<your user name>}. Please note that students in
the class are responsible for completing and submitting their own version of this assignment. While it is acceptable for
members of this class to have high-level conversations, you should not share source code or full command lines with your
classmates. Deliverables that are nearly identical to the work of others will be taken as evidence of violating the
\mbox{Honor Code}. Please see the instructor if you have questions about the policies for this assignment.

\end{document}
