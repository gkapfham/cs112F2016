% CS 111 style
% Typical usage (all UPPERCASE items are optional):
%       \input 111pre
%       \begin{document}
%       \MYTITLE{Title of document, e.g., Lab 1\\Due ...}
%       \MYHEADERS{short title}{other running head, e.g., due date}
%       \PURPOSE{Description of purpose}
%       \SUMMARY{Very short overview of assignment}
%       \DETAILS{Detailed description}
%         \SUBHEAD{if needed} ...
%         \SUBHEAD{if needed} ...
%          ...
%       \HANDIN{What to hand in and how}
%       \begin{checklist}
%       \item ...
%       \end{checklist}
% There is no need to include a "\documentstyle."
% However, there should be an "\end{document}."
%
%===========================================================
\documentclass[11pt,twoside,titlepage]{article}
%%NEED TO ADD epsf!!
\usepackage{threeparttop}
\usepackage{graphicx}
\usepackage{latexsym}
\usepackage{color}
\usepackage{listings}
\usepackage{fancyvrb}
%\usepackage{pgf,pgfarrows,pgfnodes,pgfautomata,pgfheaps,pgfshade}
\usepackage{tikz}
\usepackage[normalem]{ulem}
\tikzset{
    %Define standard arrow tip
%    >=stealth',
    %Define style for boxes
    oval/.style={
           rectangle,
           rounded corners,
           draw=black, very thick,
           text width=6.5em,
           minimum height=2em,
           text centered},
    % Define arrow style
    arr/.style={
           ->,
           thick,
           shorten <=2pt,
           shorten >=2pt,}
}
\usepackage[noend]{algorithmic}
\usepackage[noend]{algorithm}
\newcommand{\bfor}{{\bf for\ }}
\newcommand{\bthen}{{\bf then\ }}
\newcommand{\bwhile}{{\bf while\ }}
\newcommand{\btrue}{{\bf true\ }}
\newcommand{\bfalse}{{\bf false\ }}
\newcommand{\bto}{{\bf to\ }}
\newcommand{\bdo}{{\bf do\ }}
\newcommand{\bif}{{\bf if\ }}
\newcommand{\belse}{{\bf else\ }}
\newcommand{\band}{{\bf and\ }}
\newcommand{\breturn}{{\bf return\ }}
\newcommand{\mod}{{\rm mod}}
\renewcommand{\algorithmiccomment}[1]{$\rhd$ #1}
\newenvironment{checklist}{\par\noindent\hspace{-.25in}{\bf Checklist:}\renewcommand{\labelitemi}{$\Box$}%
\begin{itemize}}{\end{itemize}}
\pagestyle{threepartheadings}
\usepackage{url}
\usepackage{wrapfig}
% \usepackage{hyperref}
\usepackage[hidelinks]{hyperref}
%=========================
% One-inch margins everywhere
%=========================
\setlength{\topmargin}{0in}
\setlength{\textheight}{8.5in}
\setlength{\oddsidemargin}{0in}
\setlength{\evensidemargin}{0in}
\setlength{\textwidth}{6.5in}
%===============================
%===============================
% Macro for document title:
%===============================
\newcommand{\MYTITLE}[1]%
   {\begin{center}
     \begin{center}
     \bf
     CMPSC 111\\Introduction to Computer Science I\\
     Fall 2016\\
     \medskip
     \end{center}
     \bf
     #1
     \end{center}
}
%================================
% Macro for headings:
%================================
\newcommand{\MYHEADERS}[2]%
   {\lhead{#1}
    \rhead{#2}
    \immediate\write16{}
    \immediate\write16{DATE OF HANDOUT?}
    \read16 to \dateofhandout
    \lfoot{\sc Handed out on \dateofhandout}
    \immediate\write16{}
    \immediate\write16{HANDOUT NUMBER?}
    \read16 to\handoutnum
    \rfoot{Handout \handoutnum}
   }

%================================
% Macro for bold italic:
%================================
\newcommand{\bit}[1]{{\textit{\textbf{#1}}}}

%=========================
% Non-zero paragraph skips.
%=========================
\setlength{\parskip}{1ex}

%=========================
% Create various environments:
%=========================
\newcommand{\PURPOSE}{\par\noindent\hspace{-.25in}{\bf Purpose:\ }}
\newcommand{\SUMMARY}{\par\noindent\hspace{-.25in}{\bf Summary:\ }}
\newcommand{\DETAILS}{\par\noindent\hspace{-.25in}{\bf Details:\ }}
\newcommand{\HANDIN}{\par\noindent\hspace{-.25in}{\bf Hand in:\ }}
\newcommand{\SUBHEAD}[1]{\bigskip\par\noindent\hspace{-.1in}{\sc #1}\\}
%\newenvironment{CHECKLIST}{\begin{itemize}}{\end{itemize}}

\begin{document}
\MYTITLE{Exam 1 Study Guide \\ Delivered: Thursday, October 20, 2016 \\ Exam 1: Friday, October 27, 2016, 9:30 am}

\vspace*{-.35in}
\section*{Introduction}

This course will have its first exam on Thursday, October 27, 2016 from 9:30 to 10:45 am. The exam will be ``closed
notes'' and ``closed book'' and it will cover the following materials. Please review the ``Course Schedule'' on the web
site for the course to see the content and slides that we have covered to this date. Students may post questions about
this material to our Slack team. The questions on the examination will be drawn from the content in {\em Data Structures
and Algorithms in Java\/} (DSAAJ) by Michael T.\ Goodrich, Roberto Tamassia, and Michael H.\ Goldwasser.

\begin{itemize}

  \itemsep 0in

  \item Chapter One in DSAAJ, all sections (i.e., ``Java Primer'')

  \item Chapter Two in DSAAJ, all sections (i.e., ``Object-Oriented Design'')

  \item Chapter Three in DSAAJ, only Section 3.1 (i.e., ``Fundamental Data Structures'')

  \item Chapter Four in DSAAJ, skipping Section 4.4 (i.e., ``Algorithm Analysis'')

  \item Chapter Five in DSAAJ, skipping Section 5.6 (i.e., ``Recursion'')

  \item Using the many commands in the Linux operating system; editing in {\tt gvim}, compiling and executing
    programs in Linux; knowledge of the basic commands for using {\tt git} and Bitbucket.

  \item Your class notes and lecture slides and laboratory assignments 1 through 6.

\end{itemize}

\noindent The examination will include a mix of questions that will require you to draw and/or comment on a diagram,
write a short answer, explain and/or write a source code segment, or give and comment on a list of concepts or points.
The emphasis will be on the following list of illustrative subjects. Please note that this list is not exhaustive ---
rather it is designed to suggest representative topics.

\vspace*{-.05in}
\begin{itemize}

  \itemsep 0in

  \item Key programming constructs in the Java language (e.g., conditional logic and iteration).

  \item Declaring and using arrays and random number generators in the Java language.

  \item Object-oriented design concepts (e.g., inheritance, encapsulation, and exceptions).

  \item The tools and concepts associated with the engineering of software (e.g., build systems).

  \item Experimental and analytical evaluation of algorithms (e.g., using timers and Big-Oh notation).

  \item The use of the doubling method to understand the worst-case performance of an algorithm.

  \item The steps for performing an asymptotic analysis of an algorithm's time complexity.


  \item The use (and misuse) of recursion in Java programs that repeatedly perform an action.

  \item Practical laboratory techniques (e.g., editing, compiling, and running programs; effectively using files and
    directories; correctly using Bitbucket through the command-line {\tt git} program).

  \item Understanding Java programs (e.g., given a short, perhaps even one line, source code segment written in Java,
    understand what it does and be able to precisely describe its output).


\end{itemize}

\section*{Examination Policies}

\vspace*{-.05in}
\noindent Minimal partial credit may be awarded for the questions that require a student to write a short answer. You
are strongly encouraged to write short, precise, and correct responses to all of the questions. When you are taking the
examination, you should do so as a ``point maximizer'' who first responds to the questions that you are most likely to
answer correctly for full points. Please keep the time limitation in mind as you are absolutely required to submit the
examination at the end of the class period unless you have written permission for extra time from a member of the
Learning Commons. Students who do not submit their examination on time will have their overall point total reduced.
Please see the course instructor if you have questions about any of these policies.

\vspace*{-.15in}
\section*{Reminder Concerning the Honor Code}

\noindent Students are required to fully adhere to the Honor Code during the completion of this exam. More details about
the Allegheny College Honor Code are provided on the syllabus. Students are strongly encouraged to carefully review the
full statement of the Honor Code before taking this exam.

\noindent The following provides you with a review of Honor Code statement from the course syllabus:

The Academic Honor Program that governs the entire academic program at Allegheny College is described in the Allegheny
Academic Bulletin.  The Honor Program applies to all work that is submitted for academic credit or to meet non-credit
requirements for graduation at Allegheny College.  This includes all work assigned for this class (e.g., examinations,
laboratory assignments, and the final project).  All students who have enrolled in the College will work under the Honor
Program.  Each student who has matriculated at the College has acknowledged the following pledge:

\vspace*{-.11in}
\begin{quote}
  I hereby recognize and pledge to fulfill my responsibilities, as defined in the Honor Code, and to maintain the
  integrity of both myself and the College community as a whole.
\end{quote}
\vspace*{-.11in}

% \noindent Students who have questions about Allegheny College's Honor Code and how it applies to the completion of a
% quiz or an examination in Computer Science 112, should immediately schedule a meeting with the course instructor to
% openly discuss their questions and concerns.

\vspace*{-.2in}
\section*{Strategies for Studying}
\vspace*{-.1in}

As you study for this examination, you are encouraged to form study groups with individuals who were previously, during
a laboratory session, a member of one of your software development and empirical study teams. You can collaborate with
these individuals to ensure that you understand all of the key concepts mentioned on this study guide. Additionally,
students are encouraged to create a Slack channel that can host questions and answers that arise as you continue to
study for the test.  Even though the course instructor will try to, whenever possible, answer review questions that
students post in this channel, you are strongly encouraged to answer the questions posted by your colleagues as this
will also help you to ensure that you fully understand the material.

When studying for the test, don't forget that the Web site for our course contains mobile-ready slides that will provide
you with an overview of the key concepts that we discussed in the first modules. You can use the color scheme in the
slides to notice points where we, for instance, completed an in-class activity, discussed a key point, or made reference
to additional details available in the DSAAJ textbook. Finally, students are strongly encouraged to schedule a meeting
during the course instructor's office hours so that we can resolve any of your questions about the material and ensure
that you have the knowledge and skills necessary for doing well on this examination. Remember, while the test is taken
individually, your review for it can be done collaboratively!

\end{document}
